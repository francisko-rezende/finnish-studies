\documentclass[../päätiedosto/pää.tex]{subfiles}
% \documentclass[]{report}
% \usepackage[utf8]{inputenc}
% \usepackage[finnish]{babel}
% \usepackage[light]{kpfonts}
% \usepackage{tabularx}
% \usepackage{ltablex}
% \usepackage{color}
% \usepackage[a4paper]{geometry}
% \usepackage{titlesec}
% \usepackage{hyperref}
% \usepackage{glossaries}

% \setcounter{secnumdepth}{0}

% \addto\captionsfinnish{\renewcommand{\chaptername}{Kappale}}

% \newglossaryentry{toissapaivana}{name={toissapäivänä},description={Tänään on Lauantai, eilen oli Lauantai sitten toissapäivänä oli Torstai}}
% \newglossaryentry{eilen}{name={eilen},description={Tänään on maanantai sitten eilen oli Suununtai}}

% \makenoidxglossaries

\begin{document}

\chapter{Hei ja tervetuloa!}
\label{kap1}
\section{Sanasto}
\label{sanasto-1}
\subsection{Viikonpäivät}
\label{viikpai}

\begin{tabularx}{\textwidth}{X X}
\textbf{Mikä päivä tänään on?} & \textbf{Milloin?} \\
maanantai (ma)                       & maanantai\textbf{na}       \\
tiistai (ti)                         & tiistai\textbf{na}        \\
keskiviikko (ke)                     & keskiviikko\textbf{na}     \\
torstai (to)                         & torstai\textbf{na}         \\
perjantai (pe)                       & perjantai\textbf{na}       \\
lauantai la)                         & lauantai\textbf{na}        \\
suununtai (su)                       & suununtai\textbf{na}       \\
                                     &                   \\
viikonloppu (la ja su)               & viikonloppu\textbf{na}     \\
\end{tabularx}%
\vspace{2mm}
toissapäivänä $\rightarrow$ eilen $\rightarrow$ tänään $\rightarrow$ huomenna $\rightarrow$ ylihuomenna

\subsection{Numerot}
\label{num}
\begin{tabularx}{\textwidth}{XX}
%\textbf{Numerot} &                    \\
0       & nolla              \\
1       & yksi               \\
2       & kaksi              \\
3       & kolme              \\
4       & neljä              \\
5       & viisi              \\
6       & kuusi              \\
7       & seitsemän          \\
8       & kahdeksan          \\
9       & yhdeksän           \\
10      & kymmenen           \\
11      & yksitoista         \\
12      & kaksitoista        \\
13      & kolmetoista        \\
14      & neljätoista        \\
15      & viisitoista        \\
16      & kussitoista        \\
17      & seitsemäntoista    \\
18      & kahdeksantoista    \\
19      & yhdeksäntoista     \\
20      & kaksikymmen\textbf{tä}      \\
21      & kaksikymmentäyksi  \\
22      & kaksikymmentäkaksi \\
        &                    \\
30      & kolmekymmen\textbf{tä}      \\
40      & neljäkymmen\textbf{tä}      \\
50      & viisikymmen\textbf{tä}      \\
60      & kuusikymen\textbf{tä}       \\
70      & seitsemänkymmen\textbf{tä}  \\
80      & kahdeksänkymmen\textbf{tä}  \\
90      & yhdeksänkymmen\textbf{tä}   \\
        &                    \\
100     & sata               \\
101     & satayksi           \\
        &                    \\
200     & kaksisata\textbf{a}         \\
300     & kolmesata\textbf{a}         \\
        &                    \\
1000    & tuhat              \\
2000    & kaksituhat\textbf{ta}       \\
500000  & viisisataatuhat\textbf{ta}  \\
1000000 & miljoona           \\
2000000 & kaksimiljoona\textbf{a}    
\end{tabularx}


\section{Kielliopi}

\subsection{Persoonapronomit ja olla-verbi}

\begin{tabularx}{\textwidth}{XX}
     & olla   \\
minä & ole\textbf{n}   \\
sinä & ole\textbf{t}   \\
hän  & ole    \\
me   & ole\textbf{mme} \\
te   & ole\textbf{tte} \\
he   & o\textbf{vat}  
\end{tabularx}

\subsection{Vokaaliharmonia}
\begin{center}
\includegraphics{vokaaliharmonia.pdf}
\end{center}
\vspace{2mm}
\textbf{HUOM!}: Yhdyssanat:\\ \\*
yö\textbar vuorossa $\rightarrow$ last word is what matters.

\end{document}